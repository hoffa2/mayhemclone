%%% DOCUMENT BEGIN

\documentclass{article}

\usepackage[utf8]{inputenc}

\usepackage{geometry}
\geometry{a4paper}



%%% PACKAGES
\usepackage{lastpage}
\usepackage{graphicx}
\usepackage{booktabs} % for much better looking tables
\usepackage{array} % for better arrays (eg matrices) in maths
\usepackage{paralist} % very flexible & customisable lists (eg. enumerate/itemize, etc.)
\usepackage{verbatim} % adds environment for commenting out blocks of text & for better verbatim
\usepackage{subfig} % make it possible to include more than one captioned figure/table in a single float
\usepackage{float}


%%% HEADERS & FOOTERS
\usepackage{fancyhdr} % This should be set AFTER setting up the page geometry
\pagestyle{fancy} % options: empty , plain , fancy
\renewcommand{\headrulewidth}{0pt} % customise the layout...
\lhead{}\chead{}\rhead{}
\lfoot{}\cfoot{\thepage}\rfoot{}

%%% SECTION TITLE APPEARANCE
\usepackage{sectsty}
\allsectionsfont{\sffamily\mdseries\upshape}

%%% ToC (table of contents) APPEARANCE
\usepackage[nottoc,notlof,notlot]{tocbibind} % Put the bibliography in the ToC
\usepackage[titles,subfigure]{tocloft} % Alter the style of the Table of Contents
\renewcommand{\cftsecfont}{\rmfamily\mdseries\upshape}
\renewcommand{\cftsecpagefont}{\rmfamily\mdseries\upshape} % No bold!


%%% END Article customizations



\widowpenalties 1 10000

%%%%%%%%%%%%%%%%%%%%%%%%%
%%%          Fill in the title details                      %%%

\def \thetitle {INF-1400-Object Oriented Programming}
\def \thesubtitle {Mandatory assignment 3}
\def \theauthor {Helge Hoff \& Øystein Tveito}

%%%%%%%%%%%%%%%%%%%%%%%%%

%\pagestyle{fancy}
\pagestyle{fancyplain} % options: empty , plain , fancy
\renewcommand{\headrulewidth}{1pt} % customise the layout...
\renewcommand{\footrulewidth}{0pt}
\lhead{\fancyplain{}{\thetitle{} -- \thesubtitle{}}}\chead{}\rhead{\fancyplain{}{\theauthor{}}}
\lfoot{}\cfoot{Page {\thepage} of \pageref{LastPage}}\rfoot{}

\begin{document}

%%% TITLE PAGE

\begin{titlepage}
\begin{center}

\textsc{\\[3.5cm] \huge University of Tromsø}\\[1.5cm]

\textsc{\LARGE \thetitle}\\[0.5cm]

\textsc{\Large \thesubtitle}\\[1.5cm]

\LARGE{\theauthor} \\[0.5cm] \large{Department of Computer Science}

\vfill
{\large \today}

\end{center}
\thispagestyle{empty}
\end{titlepage}

\newpage{}


%%% TABLE OF CONTENTS

\tableofcontents


\newpage{}

%%% DOCUMENT BODY

%%% Set counter to 1
%\setcounter{page}{1}
\section{Introduction}
For this project a clone of the arcade game Mayhem will be implemented. There is two authors on this project, so the workload will be shared between.
\subsection{Technical Background}
\subsubsection{Mayhem} 
\paragraph{}

\section{Design}
\paragraph{}
This game is a two player game sharing one screen and keyboard. Each player have a set of keys associated with him, making him able to navigate and shoot his spaceship. 

\section{Implementation}
\paragraph{}

\section{Discussion}
\paragraph{}

\section{Conclusion}
\paragraph{}



\begin{thebibliography}{9}

\bibitem{boids}
  wikipedia.org,
  \emph{Boids}:
  http://en.wikipedia.org/w/index.php?title=Boids\&oldid=580365466 
  [Online; accessed 26-02-2013]
  
\bibitem{leap}
  wikipedia.org,
  \emph{Leap}:
  http://en.wikipedia.org/w/index.php?title=Leap\_Motion\&oldid=592637003
  [Online; accessed 03-03-2013]
  

\end{thebibliography}



\end{document}
